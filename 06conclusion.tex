\chapter{結論}
本研究では,
ユーザの視界に映る床に現実の床を模したバーチャル床をアニメーション重畳表示するシステムである
TravelatARを提案し,ユーザの歩行速度を制御することを目的に評価実験を行った.


実験の結果,本システムは一定時間経過後に効果が出始めること,
「5YR天然」,「5RP天然」,「5RP人工」のテクスチャを用いたバーチャル床を
後ろからユーザを追い越す動きでアニメーション重畳表示することで「ユーザの歩行速度を早くする」と「システム未使用時に比べ遅い速度感を与えること」が示唆された.


本研究ではシースルー型のHMDを用いて実験していたが,
ビデオスルー型のHMDを用いた場合ではバーチャル床の見え方が変わり,効果が変わると仮説を立てた.
だが,新型ビデオスルー型のHMDで開発を行った際,深度カメラのAPIが公開されていなかったため,実装,実験を行えなかった\cite{ocu}.
そのため,今後の課題としてビデオスルー型のHMDでの効果を確認する必要があると考える.

今後の展望として,今回仮説としてユーザの行動制御ができると考えていたリアルテクスチャは,効果が無かったが
リアルテクスチャにも動いている目安となるものを加えることで効果が強まることが考えられる.