% abstract.tex -- 論文概要
The walking movement of humans has two main aspects: the walking direction and speed. It is possible to guide people to their destinations by controlling their walking orientation externally. External control of walking speed could be effective in mitigating spatial congestion. While also maintaining people's health. Therefore, controlling walking speed is in high demand in society.


In modern society, walking speed is conducted through symbols or characters printed on the ground, signs, and voice guidance by security personnel. While this method is excellent for guiding groups of pedestrians (crowds), giving individual instructions to each pedestrian is challenging. Additionally, instructions may become invisible or inaudible during high pedestrian density.


Based on these issues, research has been conducted on manipulating walking speed by providing visual stimuli to individual pedestrians rather than placing symbols or sounds in the environment. However, there are issues when using Virtual Reality (VR) or Augmented Reality (AR) glasses in real space. There are some discrepancies in those glasses where the floor is visible through the gaps in textures during walking, leading to a decrease in the user's sense of reality (spatial consistency).


In this study, we propose a system called "travelatAR," where individual pedestrians wear AR glasses to manipulate their speed through visual stimuli presented within the glasses. In travelatAR, virtual textures similar to the appearance of the walking surface are overlaid and can be seen through AR glasses. The goal is to control the user's walking speed by animating these textures along the direction of the pedestrian's movement. The results of the experiment revealed that using textures that were visibly moving was more effective. However, we found that the system accelerates the user's walking speed when using textures with colors opposite to the real floor. Additionally, the impact of the system becomes stronger as the walking distance increases.

