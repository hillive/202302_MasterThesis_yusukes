\chapter{序論}
\section{背景}
人間の歩行運動は歩行の向きと速度の2つに分けて考えることができる.
歩行の向きを外部から操作することで人々の目的地を誘導できる.
歩行速度を外部から操作することで,人々の健康維持や人通りの多い道で誰かの歩行速度が落ちると混雑が生まれることから,歩行速度を落ちないように誘導することで空間の混雑緩和への効果が期待できる.
歩行速度を操作することは,社会的需要が高い.
\section{課題}
現代社会では,歩行速度を路面や標識に印字された記号や文字,警備員による音声案内で誘導をしている\cite{hyosiki}\cite{DJ}.
この手法は歩行者の集団(群衆)を誘導するのには優れているが,個々の歩行者へ個別の指示を伝達して誘導するのは難しい.
加えて,歩行者の密度が過密になると指示が見えない・聞こえない状態となる.
このような背景から,環境中に記号や音声を配置するのではなく,個々の歩行者に視覚刺激を与えることで歩行速度を操作する研究が実施されている.


しかし,
実空間を歩行中の被験者にヘッドマウントディスプレイを通してオプティカルフローの速度と中心視野を変化させたバーチャル空間を提示し,
被験者の歩行速度の変化を調べた研究では,オプティカルフローの速度と被験者の歩行速度の間に負の相関があると示唆された.
だが,この手法は仮想現実(VR)に基づいており,現実空間の歩行速度誘導に利用するのは難しい\cite{VR}\cite{tanizaki}.
ARグラスを通して見える床面上に「動く歩道」のテクスチャをアニメーション提示し、その際の被験者の歩行運動を観察した研究では,拡張現実(AR)に基づいており,
現実空間の歩行速度誘導への適合性は高いものの、著者らが期待した誘導効果は確認できなかった\cite{AR}\cite{sakura}.その理由として,ARで提示したテクスチャの見た目が現実の床面の見た目と差があること,
歩行の際にテクスチャの隙間から床面が見えてしまい,ユーザからみた現実感(ユーザから見える空間の整合性)が低下したことの影響を挙げられた.

\section{目的}
本研究では,個々の歩行者がARグラスを装着し,その中に提示する視覚刺激を通して歩行者の速度を操作するシステムTravelatARを提案する.
TravelatARでは,ARグラスを通してみる歩行面の上に,その見た目に近いテクスチャーを重畳表示する.
そのテクスチャーを歩行者の進行方向に沿ってアニメーションさせることでユーザの歩行速度を制御することを目的とする.
\section{本論文の構成}
第1章では研究背景,研究課題,本研究の目的について述べた.第2章では本研究の
関連研究,関連技術について述べる.第3章では本研究であるTravelatARついて述べる.
第4章では評価実験ついて述べる.第5章では第4章で行った評価実験での反省を踏まえて行った追加実験について述べる.
第6章では本研究の結論およびまとめについて述べる.
