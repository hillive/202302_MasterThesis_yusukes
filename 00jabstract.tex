% abstract.tex -- 論文概要

人間の歩行運動は歩行の向きと速度の2つに分けて考えることができる.
歩行の向きを外部から操作することで人々の目的地を誘導することができる.
歩行速度を外部から操作することで,空間の混雑緩和や人々の健康維持への効果が期待できる.
そのため,歩行速度を操作することは,社会的需要が高い.

現代社会では,歩行速度を路面や標識に印字された記号や文字,警備員による音声案内で誘導をしている.
この手法は歩行者の集団(群衆)を誘導するのには優れているが,個々の歩行者へ個別の指示を伝達して誘導するのは難しい.
加えて,歩行者の密度が過密になると指示が見えない・聞こえない状態となる.
このような背景から,環境中に記号や音声を配置するのではなく,個々の歩行者に視覚刺激を与えることで歩行速度を操作する研究が実施されている.
しかし,VRを用いた研究では現実空間の歩行速度誘導に利用するのは難しいことや,
ARグラスを用いた研究では歩行の際にテクスチャの隙間から床面が見えてしまい,
ユーザからみた現実感(ユーザから見える空間の整合性)が低下したことの影響を挙げられるなど,課題が残っている.


本研究では,個々の歩行者がARグラスを装着し,その中に提示する視覚刺激を通して歩行者の速度を操作するシステムtravelatARを提案する.
travelatARでは,ARグラスを通してみる歩行面の上に,その見た目に近いテクスチャを重畳表示する.
そのテクスチャを歩行者の進行方向に沿ってアニメーションさせることでユーザの歩行速度を制御することを目的とする.


実験の結果,テクスチャは現実の床に近似させるのではなく動いていることが分かりやすいテクスチャを用いた方が効果がでるとわかった.
しかし,本システムは前からユーザに迫ってくるように動かす際に現実の床の色と反対色のテクスチャを用いることでユーザの歩行速度を早くすること,
また,歩行距離が長くなるほど与える影響が強くなることが考察できた.
